% !TEX root = ../../Build/main.tex
% ###################################################################
% Copyright (c) 2018, Marc De Graef 
%  Editors: A.D. Rollett & M. De Graef
% All rights reserved.
%
% Licensed under the Creative Commons CC BY-NC-SA 4.0 License, 
% hereafter referred to as the "License"; you may not use this 
% document except in compliance with the License. You may obtain 
% a copy of the License at 
%     https://creativecommons.org/licenses/by-nc-sa/4.0/legalcode 
% Unless required by applicable law or agreed to in writing, all 
% material distributed under the License is distributed on an 
% "AS IS" BASIS, WITHOUT WARRANTIES OR CONDITIONS OF ANY KIND, 
% either express or implied. See the License for the specific 
% language governing permissions and limitations under the License.
% ###################################################################

% ###################################################################
% The following lines are to be uncommented or edited as needed 

\corechapter{Yes}
%     uncomment this line only if this chapter is a core/foundational chapter;
%     for a core chapter a "Core" label will appear on the top left above the chapter title.

\chapterauthor{Marc De Graef, Carnegie Mellon University}
%     this will appear in a secondary header below the chapter title.

% All figures are stored in the src/Symmetry/eps folder and must be of the *.eps type. 
\renewcommand{\chaptergraphicspath}{src/Symmetry/eps/}

\chapterimage{\noheaderimage}
%     replace \noheaderimage by the chapter header image file name (without .eps extension).
%     Chapter header images must be 2480 x 1240 pixels with 300dpi, RGB format.
\renewcommand{\chabbr}{CRYSYM}
% ###################################################################

\chapter{Crystallographic Symmetry\label{chap:Symmetry}}


\newcommand{\mbmi}[1]{$\mathbf{\mathit{#1}}$}

\section{Crystallographic Rotation Groups}
The set of special orthogonal $3\times 3$ matrices forms the rotation group $SO(3)$.  In order for a set to form a group, the set elements must satisfy four conditions under the operation of rotation composition (i.e., matrix multiplication):
\begin{enumerate}
	\item the set is \textit{closed} under composition: the product of any two rotation matrices is again a rotation matrix;
	\item the composition rule is \textit{associative}: for any three rotations $R_1$, $R_2$, and $R_3$ we must have $R_1(R_2R_3)=(R_1R_2)R_3$;
	\item there exists an \textit{identity} element: representing the Kronecker delta matrix by $\delta$, we must have for every rotation $R$ that $R\delta=\delta R=R$;
	\item each element has an \textit{inverse}: for every rotation $R$ there is an opposite rotation $R^{-1}=R^T$ such that $RR^{-1}=R^{-1}R=\delta$.
\end{enumerate}
In crystallography, the allowed rotation operations are limited to those that bring a Bravais lattice into coincidence with itself; this condition limits the possible rotation angles to $2\pi, \pi, 2\pi/3, \pi/2,$ and $\pi/3$.  The first of these corresponds to the identity rotation, so the simplest crystallographic rotation group only has the identity element and is represented by the symbol \mbmi{1}.  From among the other $31$ crystallographic point groups, we find that ten consist of only rotation operations; these are the \textit{crystallographic rotation groups},\index{crystallographic rotation groups} and they are of central importance to texture analysis.  The rotation groups are \mbmi{2}, \mbmi{3}, \mbmi{4}, \mbmi{6}, \mbmi{222}, \mbmi{32}, \mbmi{422}, \mbmi{622}, \mbmi{23}, and \mbmi{432}.  Table~\ref{tb:rotgroups} lists the \textit{generator}\index{generator} elements for each rotation group as well as the order of the group and the Sch\"onflies notation. The complete group can be generated by repeated multiplication of the generator matrices with themselves and each other until no further new rotation matrices can be found.

The groups with a single rotation axis are known as the \textit{cyclic groups}\index{cyclic groups}; the \textit{dihedral groups}\index{dihedral groups} have two-fold rotation axes normal to the principal rotation axis. There are two cubic rotation groups, namely the \textit{tetrahedral group}\index{tetrahedral group} $T$ and the \textit{octahedral group}\index{octahedral group} $O$.  Fig.~\ref{fig:rpg} shows stereographic projections of the $11$ rotation groups; 3D rendered representations are shown in Fig.~\ref{fig:rpg3D}.


\begin{table}[t]
\centering\leavevmode
\begin{tabular}{cccc}
\hline
International & Sch\"onflies & Order & Generators \\
\hline
\mbmi{1} &  $C_1$ &  $1$  &  $\delta$ \\
\mbmi{2} &  $C_2$ &  $2$  &  $\pi@[001]$ \\
\mbmi{3} &  $C_3$ &  $3$  &  $\frac{2\pi}{3}@[001]$ \\
\mbmi{4} &  $C_4$ &  $4$  &  $\frac{\pi}{4}@[001]$ \\
\mbmi{6} &  $C_6$ &  $6$  &  $\frac{\pi}{3}@[001]$ \\
\mbmi{222} &  $D_2$ &  $4$  &  $\pi@[001],\ \pi@[010]$ \\
\mbmi{32} &  $D_3$ &  $6$  &  $\frac{2\pi}{3}@[001],\ \pi@[010]$ \\
\mbmi{422} &  $D_4$ &  $8$  &  $\frac{\pi}{4}@[001],\ \pi@[010]$ \\
\mbmi{622} &  $D_6$ &  $12$  &  $\frac{\pi}{3}@[001],\ \pi@[010]$ \\
\mbmi{23} &  $T$ &  $12$  &  $\pi@[001],\ \frac{2\pi}{3}@111]$ \\
\mbmi{432} &  $O$ &  $24$  &  $\frac{\pi}{4}@[001],\ \frac{2\pi}{3}@111]$ \\
\hline
\end{tabular}
\caption{The $11$ crystallographic rotation groups, along with their Sch\"onflies symbol, group order, and group generators. {\color{red}CHECK ALL GENERATORS !!!}}
\label{tb:rotgroups}
\end{table}


